\documentclass[twocolumn, 9pt]{article}

\usepackage[utf8]{inputenc}
\usepackage{enumitem}
\usepackage{tikz}
\usepackage[utf8]{inputenc}
\usepackage[english]{babel}
\usepackage{minted}
\usepackage{amsthm}
\usepackage{xcolor}
\usepackage{hyperref}

\usetikzlibrary{arrows, patterns, positioning, shapes, automata, fit, trees, matrix}
\usetikzlibrary{arrows.meta}
\theoremstyle{plain}


\newtheorem*{conclusion}{Conclusion}

\newtheorem*{consequences}{Consequence}

\newtheorem*{definition}{Definition}

\newcommand{\code}[1]{\texttt{#1}}


\newcommand{\writecode}[3] {
\usemintedstyle{friendly}
\begin{figure}[ht]
    \centering
    \inputminted{#1}{#2}
    \caption{#3}
\end{figure}
}

\title{Simon Peyton Jones \\ Researcher Presentation}
\author{Alexandru Stoica \\
Babeș-Bolyai University \\ 
Computer Science Department 248}
\date{\today}

\addbibresource{references.bib}

\setlength{\parindent}{2em}
\setlength{\parskip}{1em}

\setlength{\columnseprule}{1pt}
\def\columnseprulecolor{\color{gray}}
\setlength{\columnsep}{1cm}

\newcommand{\writecode}[3] {
\usemintedstyle{friendly}
\begin{figure}[h!]
    \centering
    \inputminted{#1}{#2}
    \caption{#3}
\end{figure}
}

\newcommand{\code}[1]{\texttt{#1}}

\newtheorem{remark}{Utility}[subsection]
\newtheorem{corollary}{Motivation}
\newtheorem{lemma}{Assembly Connection}[subsection]

\newtheorem{observation}{Answer}[section]
\newtheorem{conclusion}{Conclusion}


\setlength\columnseprule{0pt}

\begin{document}
\maketitle


\begin{abstract}
Simon Peyton Jones is the primary contributor to Glasgow Haskell Compiler (GHS) and Haskell's design as a functional language, which re-envisions the way we write programs today. His remarkable work "The Haskell 98 Language Report" and remarks on Wadler Philip's paper "Comprehending Monads" (June 1990) helped our community to introduce monads, a mathematical concept from category theory as a standard functional design pattern. We present a summary of his personal activity and contributions as a researcher.
\end{abstract}


\section{Introduction}
Functional programming plays a central role in forming our next generation of programmers and software engineers; a fact confirmed by an increasing number of companies that are opting for functional languages \cite{tiobe:index}.

Simon Peyton Jones is a British computer scientist born on 18 January 1958 \cite{wiki:jones}, whose research shaped Haskell as a lazy functional programming language. His work as a lead developer concentrated around the Glasgow Haskell Compiler (GHC) and its ramifications \cite{wiki:jones}.

The available public information about Simon Jones's personal life is limited, published mainly by Jones himself. He is married to Dorothy Peyton Jones, a priest in the Church of England, and has six children \cite{microsoft:jones}.

In 1980, Simon Peyton Jones graduated from Trinity College, and although he never got a PhD in computer science, he became a pillar at Microsoft Research's lab in Cambridge, England \cite{jones} since 1998 \cite{wiki:jones} (Section \ref{sec::professional Activity}). During his years at Trinity College, he worked on designing and writing high-level compilers for the school's computers \cite{jones}.

He served as a lecturer at University College London and a professor at the University of Glasgow from 1990 to 1998 \cite{wiki:jones}.

His field of expertise is lazy functional programming \cite{jones}, focusing on language's design and implementation (Section \ref{sec::scientific::recognition}).

\section{Professional Activity}
\label{sec::professional Activity}

Simon Peyton Jones is a significant contributor to the design of Haskell programming language, being the editor of "The Haskell 98 Language Report" (December 2002) a document which serves as a documentation of Haskell 98 Language and Libraries \cite{haskell98}.

He co-created the C\-\- programming language (1997) which was designed to be generated by compilers for high-level languages such as Haskell \cite{wiki:jones}. 

As a vital contributor to the book Cybernauts Awake (1999), Simon Peyton Jones sought to explore the ethical and spiritual implications raised by our new technologies from within a Christian context. 

Peyton Jones is currently a chairman at Computing At School (CAS) group, an organisation that aims to promote computer science as a domain of interest in our education system \cite{wiki:jones}. He also provides educational talks, promoted by Microsoft in which he shares his insights, rather than lectures on topics such as "How to write a great research paper?" \cite{htwagp}.

\section{Scientific Recognition}
\label{sec::scientific::recognition}


At the time of this writing, Peyton Jones contributed to over 370 research papers, as a researcher at Microsoft's Research Institute in Cambridge \cite{microsoft:research:jones}; mostly focused on lazy functional programming in Haskell such as "Safe zero-cost coercions for Haskell" and "Injective Type Families for Haskell", yet some concentrated around Microsoft's technologies and products including "A User-Centred Approach to Functions in Excel" \cite{microsoft:research:jones}. Furthermore, his interests in designing useful languages are noticeable in his papers, e.g. "Backpack: Retrofitting Haskell with Interfaces" where he presents Backpack, a new language for building separately type-checkable *packages* on top of a weak module system like Haskell's \cite{backpack}. 

As a reward for his contributions to functional programming languages, in 2004 Jones was inducted as a Fellow of the Association for Computing Machinery \cite{wiki:jones}; he received a couple of years later in 2011 membership in the Academia Europaea \cite{wiki:jones}, the European Union's Academy of Humanities and Sciences.

In the same year 2011, Simon Jones and Simon Marlow were awarded for their work on GHC with the SIGPLAN Programming Languages Software Award. Two years later, he received an honorary doctorate from the University of Glasgow \cite{wiki:jones}. He was named a Fellow of the Royal Society (FRS) in 2016 furthermore a Distinguished Fellow of the British Computer Society (DFBCS) in 2017. \cite{wiki:jones}.

\printbibliography
\end{document}
