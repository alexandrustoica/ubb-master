\section{Introduction}
\label{sec::introduction}

A monad represents a mathematical design pattern which encapsulates a given value and an additional computation or effect \cite{inproceedings}. Every monadic structure respects the following three laws

\begin{equation*}
\begin{gathered}
    unit :: x \rightarrow M x \\
    value :: M x \rightarrow x \\
    bind ::  M x \rightarrow (x \rightarrow M y) \rightarrow My
\end{gathered}
\end{equation*}

which helps us create a monad $M x$ from a given value $x$, transform our monad $M x$ to another monad of type $M y$ and return the value encapsulated in our structure. \cite{inproceedings}

In functional programming, monads are used to encapsulate effects, computations which at some point in time $t$ return a value. Due to their elegant properties, monads are a good way to abstract actions which may produce side-effects such as IO operations.

In different circumstances, a new programming paradigm named aspect-oriented programming evolved around the idea of encapsulating cross-cutting concerns (computations which affect multiple modules) into aspects. 

Thus, our community had split into two groups, to try to popularise each method and promote it in each other's programming paradigm.

In this paper, we do a comparative analysis over those two approaches, providing a simplified context in which we are using them to encapsulate behaviours: 

\begin{itemize}
    
    \item We describe the generic environment within monads and advices may help us to extend our code (section \ref{sec::problem}), and we simplify our generic model 
    to provide a runnable implementation of our solutions.
    
    \item We present a monadic data structure which solves our initial problem, and we describe a couple of advantages and disadvantages when using a monad as a way to compose behaviours (section \ref{sec::solution}).
    
    \item We provide an aspect, containing a pointcut and an advice which resolves our challenging situation (section \ref{sec::aop}), and we evaluate the solution in contrast to a monadic structure. 
    
    \item We discuss related work in section \ref{sec::related::work} and provide our evaluation's  conclusions in section \ref{sec::conclusions}. 
    
\end{itemize}

