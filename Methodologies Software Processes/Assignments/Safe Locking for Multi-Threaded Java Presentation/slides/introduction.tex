\section{Introduction}


\begin{frame}
\frametitle{Introduction}
\framesubtitle{Motivation}
In Java, lexically scoped synchronized blocks represent the fundamental mechanism for concurrency control.
\\\\
For more flexible control, Java 5 introduced non-lexical operators, supporting locks primitives on re-entrant locks.

\begin{itemize}
    \item \texttt{ReentrantLock} class: lock and unlock operators
    \item transactional memory: onacid and commit operators
\end{itemize}
\end{frame}

\begin{frame}
\frametitle{Introduction}
\framesubtitle{Motivation}
The paper develops a \textbf{static type and effect system} to prevent the lock errors generated by improper use of those operators, for lock handling:
\begin{itemize}
    \item taking a lock without releasing it (which could lead to a deadlock)
    \item trying to release a lock without owning it
\end{itemize}
\end{frame}

\begin{frame}
\frametitle{Introduction}
\framesubtitle{Challenges}
The analysis needs to take into account the re-entrant lock's identity available at the program level, and keeps track of 
\begin{itemize}
    \item which lock is taken by which thread \item how many times it has been taken
\end{itemize}
\end{frame}

\begin{frame}
\frametitle{Introduction}
\framesubtitle{Challenges}
The analysis needs to handle three primary features: 
\begin{itemize}
    \item dynamic lock creation 
    \item passing of lock references
    \item aliasing of lock references
\end{itemize}
\end{frame}