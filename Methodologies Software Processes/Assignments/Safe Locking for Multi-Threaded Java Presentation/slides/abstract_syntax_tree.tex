\section{Concurrent Object-Oriented Calculus}

\begin{frame}
\frametitle{Concurrent Object-Oriented Calculus}
\framesubtitle{Featherweight Java}

\begin{definition}
Featherweight Java (FJ) is a generic name for Java-related core calculi, an object-oriented core language.
\end{definition}

The calculus used is a variant of Featherweight Java with destructive field updates, concurrency and explicit lock support, but without inheritance and type casts.
\end{frame}

\begin{frame}
\frametitle{Concurrent Object-Oriented Calculus}
\framesubtitle{Abstract Syntax Tree}
\begin{center}
  \begin{minipage}{0.8\linewidth}
    \begin{grammar}
    <D> $\coloncolonequals$ class C($\vec{f}:\vec{T}$) \{$\vec{f}:\vec{T}; \vec{M}$\}

    <M> $\coloncolonequals$ m($\vec{x}:\vec{T}$)\{t\}:T
    
    <t> $\coloncolonequals$ stop | error | v | let x:T = e in t 
    
    <e> $\coloncolonequals$ t | if v then e else e | v.f | v.f := v 
        \indalt v.m($\vec{v}$) | spawn t
        \indalt new C($\vec{v}$) | new L 
        \indalt v.lock | v.unlock
        \indalt if v trylock then e else e 
    
    <v> $\coloncolonequals$ r | x | () 
    
    <T> $\coloncolonequals$ C | B | Unit | L
    \end{grammar}
  \end{minipage}
\end{center}
\end{frame}


\begin{frame}
\frametitle{Concurrent Object-Oriented Calculus}
\framesubtitle{Abstract Syntax Tree - Thread Syntax}
\begin{center}
  \begin{minipage}{0.8\linewidth}
    \begin{grammar}
    <D> $\coloncolonequals$ class C($\vec{f}:\vec{T}$) \{$\vec{f}:\vec{T}; \vec{M}$\}

    <M> $\coloncolonequals$ m($\vec{x}:\vec{T}$)\{t\}:T
\end{grammar}
    \fcolorbox{red}{yellow}{
    \minipage[t]{\linewidth}
    \begin{grammar}
        <t> $\coloncolonequals$ stop | error | v | let x:T = e in t
    \end{grammar}
    \endminipage}\hfill
\begin{grammar}
    <e> $\coloncolonequals$ t | if v then e else e | v.f | v.f := v 
        \indalt v.m($\vec{v}$) | spawn t
        \indalt new C($\vec{v}$) | new L 
        \indalt v.lock | v.unlock
        \indalt if v trylock then e else e 
    
    <v> $\coloncolonequals$ r | x | () 
    
    <T> $\coloncolonequals$ C | B | Unit | L
    \end{grammar}
\end{minipage}
\end{center}
\end{frame}



\begin{frame}
\frametitle{Concurrent Object-Oriented Calculus}
\framesubtitle{Abstract Syntax Tree - Expression Syntax}
\begin{center}
  \begin{minipage}{0.8\linewidth}
    \fcolorbox{red}{yellow}{
    \minipage[t]{\linewidth}
    \begin{grammar}
    <e> $\coloncolonequals$ t | if v then e else e | v.f | v.f := v 
        \indalt v.m($\vec{v}$) | spawn t
        \indalt new C($\vec{v}$) | new L 
        \indalt v.lock | v.unlock
        \indalt if v trylock then e else e 
    \end{grammar}
    \endminipage}\hfill
\end{minipage}
\end{center}
\begin{itemize}
    \item \ccode{if v trylock then e else e} checks the availability of a lock v for the current thread, in which case v is taken
    \item \ccode{spawn t} starts a new thread which evaluates t in parallel with the spawning thread.
    \item \ccode{new L} dynamically creates a new thread
\end{itemize}
\end{frame}


\begin{frame}
\frametitle{Concurrent Object-Oriented Calculus}
\framesubtitle{Abstract Syntax Tree - Variable and Type Syntax}
\begin{center}
  \begin{minipage}{0.5\linewidth}
    \fcolorbox{red}{yellow}{
    \minipage[t]{\linewidth}
    \begin{grammar}
      <v> $\coloncolonequals$ r | x | () 
    \end{grammar}
    \endminipage}\hfill
\begin{itemize}
    \item $r$ = reference
    \item $x$ = variable
    \item $()$ = unit value
\end{itemize}
\end{minipage}
\end{center}

\begin{minipage}{0.8\linewidth}
    \fcolorbox{red}{yellow}{
    \minipage[t]{\linewidth}
    \begin{grammar}
    <T> $\coloncolonequals$ C | B | Unit | L
    \end{grammar}
    \endminipage}\hfill
\begin{itemize}
    \item $L$ = interface Lock from Java
    \item $B$ = basic types
    \item $C$ = class names
\end{itemize}
\end{minipage}
\end{frame}
