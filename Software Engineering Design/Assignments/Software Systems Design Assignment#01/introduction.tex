\section{Introduction}

The Glasgow Haskell Compiler (GHC) is a state-of-the-art, open source, compiler and interactive cross-platform environment for the functional language Haskell. GHC supports numerous extensions, libraries, and optimisations that streamline the process of generating and executing code.


\subsection{History \& Evaluation}

GHC started as a part of an academic research project funded by the United Kingdom government at the beginning of the 90s. For over 20 years, the compiler has been under continuous active development.

As of March 2019, GHC has released eight stable major versions. Our paper will focus its analysis on the latest two versions GHC 8.x.y and GHC 7.x.y. 

GHC 8.0.1 introduced support for record pattern synonyms, injective type families, applicative do notation, the \texttt{-XDeriveAnyClass} extension, support for wildcards in data and type family instances and many other notable features.

Even thought GHC's latest version implements a couple of exciting extensions which help the programmer write with ease generic chunks of Haskell code, GHC 7 represents by far one of the most critical updates introducing \texttt{Haskell2010}

GHC 7.0.1 includes an LLVM code generator which for specific code, particularly heavy arithmetic code, can bring some nice performance improvements.

\subsection{Technologies}

The prototype of GHC was built using \texttt{LML} (Lazy ML), yet as most programming languages, today, the compiler is mostly written in Haskell, C and \texttt{C--}.