\section{Introduction}

The Visualisation Toolkit (VTK) represents a useful software system for processing and visualising data. It helps the scientific computing, medical and mathematical communities to analyse large amounts of data \cite{aosabook}.

Data visualisation includes the process of transforming data into sensory input, images in visual, tactile and auditory forms. VTK is capable of two types of data visualisation:

\begin{itemize}
    \item scientific visualisation representing spatial-temporal information \cite{aosabook}.
    \item information visualisation representing abstract data forms such as web pages, documents, unstructured pages, tables, graphs and trees \cite{aosabook}.
\end{itemize}

One of the core requirements of VTK is its ability to create data flow pipelines that are capable of processing, representing, rendering and caching data. The toolkit requires a flexible architecture at multiple levels of its representation; the principle of composition and interchangeable components are the supporting pillars of VTK.

The central content of this report focuses on the architectural and structural design patterns that make VTK a successful system for data processing.